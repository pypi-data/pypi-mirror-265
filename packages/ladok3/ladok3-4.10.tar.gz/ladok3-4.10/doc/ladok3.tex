\documentclass[a4paper,oneside]{book}
\newenvironment{abstract}{}{}
\usepackage{abstract}
\usepackage{noweb}
% Needed to relax penalty for breaking code chunks across pages, otherwise 
% there might be a lot of space following a code chunk.
\def\nwendcode{\endtrivlist \endgroup}
\let\nwdocspar=\smallbreak

\usepackage[hyphens]{url}
\usepackage{hyperref}
\usepackage{authblk}

\usepackage[utf8]{inputenc}
\usepackage[T1]{fontenc}
\usepackage[swedish,british]{babel}
\usepackage{booktabs}

\usepackage[natbib,style=alphabetic,maxbibnames=99]{biblatex}
\addbibresource{ladok_lib.bib}

\usepackage[inline]{enumitem}
\setlist[enumerate]{label=(\arabic*)}

\usepackage[all]{foreign}
\renewcommand{\foreignfullfont}{}
\renewcommand{\foreignabbrfont}{}

\usepackage{newclude}
\usepackage{import}

\usepackage[strict]{csquotes}
\usepackage[single]{acro}

\usepackage{subcaption}

\usepackage[noend]{algpseudocode}
\usepackage{xparse}

\let\email\texttt

\usepackage[outputdir=ltxobj]{minted}
\setminted{autogobble,linenos}

\usepackage{pythontex}
\setpythontexoutputdir{.}
\setpythontexworkingdir{..}

\usepackage{fancyvrb}

\usepackage{amsmath}
\usepackage{amssymb}
\usepackage{mathtools}
\usepackage{amsthm}
\usepackage{thmtools}
\usepackage[unq]{unique}
\DeclareMathOperator{\powerset}{\mathcal{P}}

\usepackage[binary-units]{siunitx}

\usepackage{pgf}

\usepackage[capitalize]{cleveref}
\crefalias{question}{item}
\crefname{question}{question}{questions}
\Crefname{question}{Question}{Questions}
\creflabelformat{question}{#2\textup{#1}#3}



\title{%
  A LADOK3 Python API
}
\author{%
  Alexander Baltatzis,
  Daniel Bosk,
  Gerald Q.\ \enquote{Chip} Maguire Jr.
}
\affil{%
  KTH EECS\\
  \texttt{\{alba,dbosk,maguire\}@kth.se}
}

\begin{document}
\frontmatter
\maketitle

\vspace*{\fill}
\VerbatimInput{../LICENSE}
\clearpage

\begin{abstract}
  We provide a Python wrapper for the LADOK3 REST API\@.
We provide a useful object-oriented framework and direct API calls that return 
the unprocessed JSON data from LADOK.

\end{abstract}
\clearpage

\tableofcontents
\clearpage

\mainmatter
\chapter{Introduction}

LADOK (abbreviation of \foreignlanguage{swedish}{Lokalt Adb–baserat 
DOKumentationssystem}, in Swedish) is the national documentation system for 
higher education in Sweden.
This is the documented source code of \texttt{ladok3}, a LADOK3 API wrapper for 
Python.

The \texttt{ladok3} library provides a non-GUI application that, similar to 
access via a web browser, only provides the user with access to the LADOK data 
and functionality that they would actually have based on their specific user 
permissions in LADOK.
It can be seem as a very streamlined web browser just for LADOK's web 
interface.
While the library exploits caching to reduce the load on the LADOK server, this 
represents a subset of the information that would otherwise be obtained via
LADOK's web GUI export functions.

You can install the \texttt{ladok3} package by running
\begin{minted}{bash}
pip3 install ladok3
\end{minted}
in the terminal.
You can find a quick reference by running
\begin{minted}[firstnumber=last]{bash}
pydoc ladok3
\end{minted}

We provide the main class \texttt{LadokSession} (\cref{LadokSession}).
The \texttt{LadokSession} class acts like \enquote{factories} and will return 
objects representing various LADOK data.
These data objects' classes inherit the \texttt{LadokData} (\cref{LadokData}) 
and \texttt{LadokRemoteData} (\cref{LadokRemoteData}) classes.
Data of the type \texttt{LadokData} is not expected to change, unlike 
\texttt{LadokRemoteData}, which is.
Objects of type \texttt{LadokRemoteData} know how to update themselves, \ie fetch 
and refresh their data from LADOK.
When relevant they can also write data to LADOK, \ie update entries such as 
results.

One design criteria is to improve performance.
We do this by caching all factory methods of any \texttt{LadokSession}.
The \texttt{LadokSession} itself is also designed to be cacheable: if the session to 
LADOK expires, it will automatically reauthenticate to establish a new session.



\part{The library}

\documentclass[a4paper,oneside]{book}
\newenvironment{abstract}{}{}
\usepackage{abstract}
\usepackage{noweb}
% Needed to relax penalty for breaking code chunks across pages, otherwise 
% there might be a lot of space following a code chunk.
\def\nwendcode{\endtrivlist \endgroup}
\let\nwdocspar=\smallbreak

\usepackage[hyphens]{url}
\usepackage{hyperref}
\usepackage{authblk}

\usepackage[utf8]{inputenc}
\usepackage[T1]{fontenc}
\usepackage[swedish,british]{babel}
\usepackage{booktabs}

\usepackage[natbib,style=alphabetic,maxbibnames=99]{biblatex}
\addbibresource{ladok_lib.bib}

\usepackage[inline]{enumitem}
\setlist[enumerate]{label=(\arabic*)}

\usepackage[all]{foreign}
\renewcommand{\foreignfullfont}{}
\renewcommand{\foreignabbrfont}{}

\usepackage{newclude}
\usepackage{import}

\usepackage[strict]{csquotes}
\usepackage[single]{acro}

\usepackage{subcaption}

\usepackage[noend]{algpseudocode}
\usepackage{xparse}

\let\email\texttt

\usepackage[outputdir=ltxobj]{minted}
\setminted{autogobble,linenos}

\usepackage{pythontex}
\setpythontexoutputdir{.}
\setpythontexworkingdir{..}

\usepackage{fancyvrb}

\usepackage{amsmath}
\usepackage{amssymb}
\usepackage{mathtools}
\usepackage{amsthm}
\usepackage{thmtools}
\usepackage[unq]{unique}
\DeclareMathOperator{\powerset}{\mathcal{P}}

\usepackage[binary-units]{siunitx}

\usepackage{pgf}

\usepackage[capitalize]{cleveref}
\crefalias{question}{item}
\crefname{question}{question}{questions}
\Crefname{question}{Question}{Questions}
\creflabelformat{question}{#2\textup{#1}#3}



\title{%
  A LADOK3 Python API
}
\author{%
  Alexander Baltatzis,
  Daniel Bosk,
  Gerald Q.\ \enquote{Chip} Maguire Jr.
}
\affil{%
  KTH EECS\\
  \texttt{\{alba,dbosk,maguire\}@kth.se}
}

\begin{document}
\frontmatter
\maketitle

\vspace*{\fill}
\VerbatimInput{../LICENSE}
\clearpage

\begin{abstract}
  We provide a Python wrapper for the LADOK3 REST API\@.
We provide a useful object-oriented framework and direct API calls that return 
the unprocessed JSON data from LADOK.

\end{abstract}
\clearpage

\tableofcontents
\clearpage

\mainmatter
\chapter{Introduction}

LADOK (abbreviation of \foreignlanguage{swedish}{Lokalt Adb–baserat 
DOKumentationssystem}, in Swedish) is the national documentation system for 
higher education in Sweden.
This is the documented source code of \texttt{ladok3}, a LADOK3 API wrapper for 
Python.

The \texttt{ladok3} library provides a non-GUI application that, similar to 
access via a web browser, only provides the user with access to the LADOK data 
and functionality that they would actually have based on their specific user 
permissions in LADOK.
It can be seem as a very streamlined web browser just for LADOK's web 
interface.
While the library exploits caching to reduce the load on the LADOK server, this 
represents a subset of the information that would otherwise be obtained via
LADOK's web GUI export functions.

You can install the \texttt{ladok3} package by running
\begin{minted}{bash}
pip3 install ladok3
\end{minted}
in the terminal.
You can find a quick reference by running
\begin{minted}[firstnumber=last]{bash}
pydoc ladok3
\end{minted}

We provide the main class \texttt{LadokSession} (\cref{LadokSession}).
The \texttt{LadokSession} class acts like \enquote{factories} and will return 
objects representing various LADOK data.
These data objects' classes inherit the \texttt{LadokData} (\cref{LadokData}) 
and \texttt{LadokRemoteData} (\cref{LadokRemoteData}) classes.
Data of the type \texttt{LadokData} is not expected to change, unlike 
\texttt{LadokRemoteData}, which is.
Objects of type \texttt{LadokRemoteData} know how to update themselves, \ie fetch 
and refresh their data from LADOK.
When relevant they can also write data to LADOK, \ie update entries such as 
results.

One design criteria is to improve performance.
We do this by caching all factory methods of any \texttt{LadokSession}.
The \texttt{LadokSession} itself is also designed to be cacheable: if the session to 
LADOK expires, it will automatically reauthenticate to establish a new session.



\part{The library}

\documentclass[a4paper,oneside]{book}
\newenvironment{abstract}{}{}
\usepackage{abstract}
\usepackage{noweb}
% Needed to relax penalty for breaking code chunks across pages, otherwise 
% there might be a lot of space following a code chunk.
\def\nwendcode{\endtrivlist \endgroup}
\let\nwdocspar=\smallbreak

\usepackage[hyphens]{url}
\usepackage{hyperref}
\usepackage{authblk}

\usepackage[utf8]{inputenc}
\usepackage[T1]{fontenc}
\usepackage[swedish,british]{babel}
\usepackage{booktabs}

\usepackage[natbib,style=alphabetic,maxbibnames=99]{biblatex}
\addbibresource{ladok_lib.bib}

\usepackage[inline]{enumitem}
\setlist[enumerate]{label=(\arabic*)}

\usepackage[all]{foreign}
\renewcommand{\foreignfullfont}{}
\renewcommand{\foreignabbrfont}{}

\usepackage{newclude}
\usepackage{import}

\usepackage[strict]{csquotes}
\usepackage[single]{acro}

\usepackage{subcaption}

\usepackage[noend]{algpseudocode}
\usepackage{xparse}

\let\email\texttt

\usepackage[outputdir=ltxobj]{minted}
\setminted{autogobble,linenos}

\usepackage{pythontex}
\setpythontexoutputdir{.}
\setpythontexworkingdir{..}

\usepackage{fancyvrb}

\usepackage{amsmath}
\usepackage{amssymb}
\usepackage{mathtools}
\usepackage{amsthm}
\usepackage{thmtools}
\usepackage[unq]{unique}
\DeclareMathOperator{\powerset}{\mathcal{P}}

\usepackage[binary-units]{siunitx}

\usepackage{pgf}

\usepackage[capitalize]{cleveref}
\crefalias{question}{item}
\crefname{question}{question}{questions}
\Crefname{question}{Question}{Questions}
\creflabelformat{question}{#2\textup{#1}#3}



\title{%
  A LADOK3 Python API
}
\author{%
  Alexander Baltatzis,
  Daniel Bosk,
  Gerald Q.\ \enquote{Chip} Maguire Jr.
}
\affil{%
  KTH EECS\\
  \texttt{\{alba,dbosk,maguire\}@kth.se}
}

\begin{document}
\frontmatter
\maketitle

\vspace*{\fill}
\VerbatimInput{../LICENSE}
\clearpage

\begin{abstract}
  We provide a Python wrapper for the LADOK3 REST API\@.
We provide a useful object-oriented framework and direct API calls that return 
the unprocessed JSON data from LADOK.

\end{abstract}
\clearpage

\tableofcontents
\clearpage

\mainmatter
\chapter{Introduction}

LADOK (abbreviation of \foreignlanguage{swedish}{Lokalt Adb–baserat 
DOKumentationssystem}, in Swedish) is the national documentation system for 
higher education in Sweden.
This is the documented source code of \texttt{ladok3}, a LADOK3 API wrapper for 
Python.

The \texttt{ladok3} library provides a non-GUI application that, similar to 
access via a web browser, only provides the user with access to the LADOK data 
and functionality that they would actually have based on their specific user 
permissions in LADOK.
It can be seem as a very streamlined web browser just for LADOK's web 
interface.
While the library exploits caching to reduce the load on the LADOK server, this 
represents a subset of the information that would otherwise be obtained via
LADOK's web GUI export functions.

You can install the \texttt{ladok3} package by running
\begin{minted}{bash}
pip3 install ladok3
\end{minted}
in the terminal.
You can find a quick reference by running
\begin{minted}[firstnumber=last]{bash}
pydoc ladok3
\end{minted}

We provide the main class \texttt{LadokSession} (\cref{LadokSession}).
The \texttt{LadokSession} class acts like \enquote{factories} and will return 
objects representing various LADOK data.
These data objects' classes inherit the \texttt{LadokData} (\cref{LadokData}) 
and \texttt{LadokRemoteData} (\cref{LadokRemoteData}) classes.
Data of the type \texttt{LadokData} is not expected to change, unlike 
\texttt{LadokRemoteData}, which is.
Objects of type \texttt{LadokRemoteData} know how to update themselves, \ie fetch 
and refresh their data from LADOK.
When relevant they can also write data to LADOK, \ie update entries such as 
results.

One design criteria is to improve performance.
We do this by caching all factory methods of any \texttt{LadokSession}.
The \texttt{LadokSession} itself is also designed to be cacheable: if the session to 
LADOK expires, it will automatically reauthenticate to establish a new session.



\part{The library}

\documentclass[a4paper,oneside]{book}
\newenvironment{abstract}{}{}
\usepackage{abstract}
\usepackage{noweb}
% Needed to relax penalty for breaking code chunks across pages, otherwise 
% there might be a lot of space following a code chunk.
\def\nwendcode{\endtrivlist \endgroup}
\let\nwdocspar=\smallbreak

\usepackage[hyphens]{url}
\usepackage{hyperref}
\usepackage{authblk}

\input{preamble.tex}

\title{%
  A LADOK3 Python API
}
\author{%
  Alexander Baltatzis,
  Daniel Bosk,
  Gerald Q.\ \enquote{Chip} Maguire Jr.
}
\affil{%
  KTH EECS\\
  \texttt{\{alba,dbosk,maguire\}@kth.se}
}

\begin{document}
\frontmatter
\maketitle

\vspace*{\fill}
\VerbatimInput{../LICENSE}
\clearpage

\begin{abstract}
  \input{abstract.tex}
\end{abstract}
\clearpage

\tableofcontents
\clearpage

\mainmatter
\chapter{Introduction}

LADOK (abbreviation of \foreignlanguage{swedish}{Lokalt Adb–baserat 
DOKumentationssystem}, in Swedish) is the national documentation system for 
higher education in Sweden.
This is the documented source code of \texttt{ladok3}, a LADOK3 API wrapper for 
Python.

The \texttt{ladok3} library provides a non-GUI application that, similar to 
access via a web browser, only provides the user with access to the LADOK data 
and functionality that they would actually have based on their specific user 
permissions in LADOK.
It can be seem as a very streamlined web browser just for LADOK's web 
interface.
While the library exploits caching to reduce the load on the LADOK server, this 
represents a subset of the information that would otherwise be obtained via
LADOK's web GUI export functions.

You can install the \texttt{ladok3} package by running
\begin{minted}{bash}
pip3 install ladok3
\end{minted}
in the terminal.
You can find a quick reference by running
\begin{minted}[firstnumber=last]{bash}
pydoc ladok3
\end{minted}

We provide the main class \texttt{LadokSession} (\cref{LadokSession}).
The \texttt{LadokSession} class acts like \enquote{factories} and will return 
objects representing various LADOK data.
These data objects' classes inherit the \texttt{LadokData} (\cref{LadokData}) 
and \texttt{LadokRemoteData} (\cref{LadokRemoteData}) classes.
Data of the type \texttt{LadokData} is not expected to change, unlike 
\texttt{LadokRemoteData}, which is.
Objects of type \texttt{LadokRemoteData} know how to update themselves, \ie fetch 
and refresh their data from LADOK.
When relevant they can also write data to LADOK, \ie update entries such as 
results.

One design criteria is to improve performance.
We do this by caching all factory methods of any \texttt{LadokSession}.
The \texttt{LadokSession} itself is also designed to be cacheable: if the session to 
LADOK expires, it will automatically reauthenticate to establish a new session.



\part{The library}

\input{../src/ladok3/ladok3.tex}


\part{API calls}

\input{../src/ladok3/api.tex}
\input{../src/ladok3/undoc.tex}



\part{A command-line interface}

\chapter{The base interface}

\input{../src/ladok3/cli.tex}

\chapter{The \texttt{data} command}

\input{../src/ladok3/data.tex}

\chapter{The \texttt{report} command}

\input{../src/ladok3/report.tex}

\chapter{The \texttt{student} command}

\input{../src/ladok3/student.tex}



\part{Other example applications}

\chapter{Transfer results from KTH Canvas to LADOK}

Here we provide an example program~\texttt{canvas2ladok.py} which exports 
results from KTH Canvas to LADOK for the introductory programming course~prgi 
(DD1315).
You can find an up-to-date version of this chapter at
\begin{center}
  \url{https://github.com/dbosk/intropy/tree/master/adm/reporting}.
\end{center}

\input{../examples/canvas2ladok.tex}


\backmatter
\printbibliography


\end{document}



\part{API calls}

\input{../src/ladok3/api.tex}
\input{../src/ladok3/undoc.tex}



\part{A command-line interface}

\chapter{The base interface}

\input{../src/ladok3/cli.tex}

\chapter{The \texttt{data} command}

\makeatletter
\tikzset{
    database/.style={
        path picture={
            \draw (0, 1.5*\database@segmentheight) circle [x radius=\database@radius,y radius=\database@aspectratio*\database@radius];
            \draw (-\database@radius, 0.5*\database@segmentheight) arc [start angle=180,end angle=360,x radius=\database@radius, y radius=\database@aspectratio*\database@radius];
            \draw (-\database@radius,-0.5*\database@segmentheight) arc [start angle=180,end angle=360,x radius=\database@radius, y radius=\database@aspectratio*\database@radius];
            \draw (-\database@radius,1.5*\database@segmentheight) -- ++(0,-3*\database@segmentheight) arc [start angle=180,end angle=360,x radius=\database@radius, y radius=\database@aspectratio*\database@radius] -- ++(0,3*\database@segmentheight);
        },
        minimum width=2*\database@radius + \pgflinewidth,
        minimum height=3*\database@segmentheight + 2*\database@aspectratio*\database@radius + \pgflinewidth,
    },
    database segment height/.store in=\database@segmentheight,
    database radius/.store in=\database@radius,
    database aspect ratio/.store in=\database@aspectratio,
    database segment height=0.1cm,
    database radius=0.25cm,
    database aspect ratio=0.35,
}
\makeatother


\chapter{The \texttt{report} command}

\documentclass[12pt,english]{article}

\begin{document}

\begin{center}
{\large Approaches to duplicated computation in Clawpack Riemann solvers} \\
David Ketcheson \\
Nov. 1, 2011
\par\end{center}{\large \par}


In some Riemann solvers, intermediate values must be computed in {\tt rpn2}
that are also needed in {\tt rpt2}; for example, the Roe average values in
Roe solvers.  This might be handled in various ways:
\begin{enumerate}
  \item Use a common block to store them.
  \item Recompute them in rpt2, allocating temporary arrays in each call of rpn2 and rpt2.
  \item Recompute them in rpt2, using pre-allocated arrays (aux or module arrays) for storage.
  \item Recompute them in the same loop as the rest of the solve, avoiding the need for storage.
\end{enumerate}
These have been implemented in the following Riemann solvers, respectively (and in the
corresponding transverse solvers):
\begin{enumerate}
  \item \verb=rpn2_euler_5wave.f=
  \item \verb=rpn2_euler_5wave_recompute.f=
  \item \verb=rpn2_euler_5wave_aux.f=
  \item \verb=rpn2_euler_5wave_rec_loc.f=
\end{enumerate}
Each was tested on the Shock-Bubble problem with a $320 \times 80$ grid.  The timings for each
were (first time is on a Macbook Air, second on a Mac Pro workstation):
\begin{enumerate}
  \item 3:45 / 1:58
  \item 4:05 / 2:10
  \item 4:04
  \item 4:04 / 2:10
\end{enumerate}
These results indicate that storing these values is worthwhile (saves nearly 10\% of the
computational cost).  Also, allocating 1D-slice size arrays in every Riemann solver call
seems to have a negligible cost in 2D.

Some things that might be worth investigating are:
\begin{enumerate}
  \item How do these results change if optimization flags are thrown?
  \item How do these results change if larger grids are used?
\end{enumerate}
%Each was tested on the Shock-Bubble problem with a $160 \times 40$ grid.  The timings for each
%were (first time is on a Macbook Air, second on a Mac Pro workstation):
%\begin{enumerate}
%  \item / 12.5
%  \item / 
%  \item 
%  \item / 13.6
%\end{enumerate}
%
%Each was tested on the Shock-Bubble problem with a $640 \times 160$ grid.  The timings for each
%were (first time is on a Macbook Air, second on a Mac Pro workstation):
%\begin{enumerate}
%  \item / 
%  \item / 
%  \item 
%  \item / 
%\end{enumerate}


\end{document}


\chapter{The \texttt{student} command}

\input{../src/ladok3/student.tex}



\part{Other example applications}

\chapter{Transfer results from KTH Canvas to LADOK}

Here we provide an example program~\texttt{canvas2ladok.py} which exports 
results from KTH Canvas to LADOK for the introductory programming course~prgi 
(DD1315).
You can find an up-to-date version of this chapter at
\begin{center}
  \url{https://github.com/dbosk/intropy/tree/master/adm/reporting}.
\end{center}

\input{../examples/canvas2ladok.tex}


\backmatter
\printbibliography


\end{document}



\part{API calls}

\input{../src/ladok3/api.tex}
\input{../src/ladok3/undoc.tex}



\part{A command-line interface}

\chapter{The base interface}

\input{../src/ladok3/cli.tex}

\chapter{The \texttt{data} command}

\makeatletter
\tikzset{
    database/.style={
        path picture={
            \draw (0, 1.5*\database@segmentheight) circle [x radius=\database@radius,y radius=\database@aspectratio*\database@radius];
            \draw (-\database@radius, 0.5*\database@segmentheight) arc [start angle=180,end angle=360,x radius=\database@radius, y radius=\database@aspectratio*\database@radius];
            \draw (-\database@radius,-0.5*\database@segmentheight) arc [start angle=180,end angle=360,x radius=\database@radius, y radius=\database@aspectratio*\database@radius];
            \draw (-\database@radius,1.5*\database@segmentheight) -- ++(0,-3*\database@segmentheight) arc [start angle=180,end angle=360,x radius=\database@radius, y radius=\database@aspectratio*\database@radius] -- ++(0,3*\database@segmentheight);
        },
        minimum width=2*\database@radius + \pgflinewidth,
        minimum height=3*\database@segmentheight + 2*\database@aspectratio*\database@radius + \pgflinewidth,
    },
    database segment height/.store in=\database@segmentheight,
    database radius/.store in=\database@radius,
    database aspect ratio/.store in=\database@aspectratio,
    database segment height=0.1cm,
    database radius=0.25cm,
    database aspect ratio=0.35,
}
\makeatother


\chapter{The \texttt{report} command}

\documentclass[12pt,english]{article}

\begin{document}

\begin{center}
{\large Approaches to duplicated computation in Clawpack Riemann solvers} \\
David Ketcheson \\
Nov. 1, 2011
\par\end{center}{\large \par}


In some Riemann solvers, intermediate values must be computed in {\tt rpn2}
that are also needed in {\tt rpt2}; for example, the Roe average values in
Roe solvers.  This might be handled in various ways:
\begin{enumerate}
  \item Use a common block to store them.
  \item Recompute them in rpt2, allocating temporary arrays in each call of rpn2 and rpt2.
  \item Recompute them in rpt2, using pre-allocated arrays (aux or module arrays) for storage.
  \item Recompute them in the same loop as the rest of the solve, avoiding the need for storage.
\end{enumerate}
These have been implemented in the following Riemann solvers, respectively (and in the
corresponding transverse solvers):
\begin{enumerate}
  \item \verb=rpn2_euler_5wave.f=
  \item \verb=rpn2_euler_5wave_recompute.f=
  \item \verb=rpn2_euler_5wave_aux.f=
  \item \verb=rpn2_euler_5wave_rec_loc.f=
\end{enumerate}
Each was tested on the Shock-Bubble problem with a $320 \times 80$ grid.  The timings for each
were (first time is on a Macbook Air, second on a Mac Pro workstation):
\begin{enumerate}
  \item 3:45 / 1:58
  \item 4:05 / 2:10
  \item 4:04
  \item 4:04 / 2:10
\end{enumerate}
These results indicate that storing these values is worthwhile (saves nearly 10\% of the
computational cost).  Also, allocating 1D-slice size arrays in every Riemann solver call
seems to have a negligible cost in 2D.

Some things that might be worth investigating are:
\begin{enumerate}
  \item How do these results change if optimization flags are thrown?
  \item How do these results change if larger grids are used?
\end{enumerate}
%Each was tested on the Shock-Bubble problem with a $160 \times 40$ grid.  The timings for each
%were (first time is on a Macbook Air, second on a Mac Pro workstation):
%\begin{enumerate}
%  \item / 12.5
%  \item / 
%  \item 
%  \item / 13.6
%\end{enumerate}
%
%Each was tested on the Shock-Bubble problem with a $640 \times 160$ grid.  The timings for each
%were (first time is on a Macbook Air, second on a Mac Pro workstation):
%\begin{enumerate}
%  \item / 
%  \item / 
%  \item 
%  \item / 
%\end{enumerate}


\end{document}


\chapter{The \texttt{student} command}

\input{../src/ladok3/student.tex}



\part{Other example applications}

\chapter{Transfer results from KTH Canvas to LADOK}

Here we provide an example program~\texttt{canvas2ladok.py} which exports 
results from KTH Canvas to LADOK for the introductory programming course~prgi 
(DD1315).
You can find an up-to-date version of this chapter at
\begin{center}
  \url{https://github.com/dbosk/intropy/tree/master/adm/reporting}.
\end{center}

\input{../examples/canvas2ladok.tex}


\backmatter
\printbibliography


\end{document}



\part{API calls}

\input{../src/ladok3/api.tex}
\input{../src/ladok3/undoc.tex}



\part{A command-line interface}

\chapter{The base interface}

\input{../src/ladok3/cli.tex}

\chapter{The \texttt{data} command}

\makeatletter
\tikzset{
    database/.style={
        path picture={
            \draw (0, 1.5*\database@segmentheight) circle [x radius=\database@radius,y radius=\database@aspectratio*\database@radius];
            \draw (-\database@radius, 0.5*\database@segmentheight) arc [start angle=180,end angle=360,x radius=\database@radius, y radius=\database@aspectratio*\database@radius];
            \draw (-\database@radius,-0.5*\database@segmentheight) arc [start angle=180,end angle=360,x radius=\database@radius, y radius=\database@aspectratio*\database@radius];
            \draw (-\database@radius,1.5*\database@segmentheight) -- ++(0,-3*\database@segmentheight) arc [start angle=180,end angle=360,x radius=\database@radius, y radius=\database@aspectratio*\database@radius] -- ++(0,3*\database@segmentheight);
        },
        minimum width=2*\database@radius + \pgflinewidth,
        minimum height=3*\database@segmentheight + 2*\database@aspectratio*\database@radius + \pgflinewidth,
    },
    database segment height/.store in=\database@segmentheight,
    database radius/.store in=\database@radius,
    database aspect ratio/.store in=\database@aspectratio,
    database segment height=0.1cm,
    database radius=0.25cm,
    database aspect ratio=0.35,
}
\makeatother


\chapter{The \texttt{report} command}

\documentclass[12pt,english]{article}

\begin{document}

\begin{center}
{\large Approaches to duplicated computation in Clawpack Riemann solvers} \\
David Ketcheson \\
Nov. 1, 2011
\par\end{center}{\large \par}


In some Riemann solvers, intermediate values must be computed in {\tt rpn2}
that are also needed in {\tt rpt2}; for example, the Roe average values in
Roe solvers.  This might be handled in various ways:
\begin{enumerate}
  \item Use a common block to store them.
  \item Recompute them in rpt2, allocating temporary arrays in each call of rpn2 and rpt2.
  \item Recompute them in rpt2, using pre-allocated arrays (aux or module arrays) for storage.
  \item Recompute them in the same loop as the rest of the solve, avoiding the need for storage.
\end{enumerate}
These have been implemented in the following Riemann solvers, respectively (and in the
corresponding transverse solvers):
\begin{enumerate}
  \item \verb=rpn2_euler_5wave.f=
  \item \verb=rpn2_euler_5wave_recompute.f=
  \item \verb=rpn2_euler_5wave_aux.f=
  \item \verb=rpn2_euler_5wave_rec_loc.f=
\end{enumerate}
Each was tested on the Shock-Bubble problem with a $320 \times 80$ grid.  The timings for each
were (first time is on a Macbook Air, second on a Mac Pro workstation):
\begin{enumerate}
  \item 3:45 / 1:58
  \item 4:05 / 2:10
  \item 4:04
  \item 4:04 / 2:10
\end{enumerate}
These results indicate that storing these values is worthwhile (saves nearly 10\% of the
computational cost).  Also, allocating 1D-slice size arrays in every Riemann solver call
seems to have a negligible cost in 2D.

Some things that might be worth investigating are:
\begin{enumerate}
  \item How do these results change if optimization flags are thrown?
  \item How do these results change if larger grids are used?
\end{enumerate}
%Each was tested on the Shock-Bubble problem with a $160 \times 40$ grid.  The timings for each
%were (first time is on a Macbook Air, second on a Mac Pro workstation):
%\begin{enumerate}
%  \item / 12.5
%  \item / 
%  \item 
%  \item / 13.6
%\end{enumerate}
%
%Each was tested on the Shock-Bubble problem with a $640 \times 160$ grid.  The timings for each
%were (first time is on a Macbook Air, second on a Mac Pro workstation):
%\begin{enumerate}
%  \item / 
%  \item / 
%  \item 
%  \item / 
%\end{enumerate}


\end{document}


\chapter{The \texttt{student} command}

\input{../src/ladok3/student.tex}



\part{Other example applications}

\chapter{Transfer results from KTH Canvas to LADOK}

Here we provide an example program~\texttt{canvas2ladok.py} which exports 
results from KTH Canvas to LADOK for the introductory programming course~prgi 
(DD1315).
You can find an up-to-date version of this chapter at
\begin{center}
  \url{https://github.com/dbosk/intropy/tree/master/adm/reporting}.
\end{center}

\input{../examples/canvas2ladok.tex}


\backmatter
\printbibliography


\end{document}
