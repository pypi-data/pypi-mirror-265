
\documentclass{article}
\usepackage{feynmp}
\usepackage[paperwidth=70cm,paperheight=120cm,centering,textwidth=70cm,textheight=100cm,left=-2cm,top=10cm]{geometry}
\usepackage{amsmath}
\usepackage{xcolor}

\thispagestyle{empty}

% Definition of constants 
\newcommand{\arcMaxDistance}{1.00}
\newcommand{\externalLegTension}{3.00}
\newcommand{\defaultTension}{1.00}
\newcommand{\lineWidth}{1.75}
\newcommand{\lineColor}{black}
\newcommand{\lmbColor}{red}
\newcommand{\labelColor}{black}
\newcommand{\labelSize}{20pt}
\newcommand{\nonLmbColor}{blue}
\newcommand{\arrowSize}{1.50}
\newcommand{\doubleArrowSize}{2.00}
\newcommand{\labelDistance}{13.00}
\newcommand{\anchorTension}{2.00}
\newcommand{\anchorLine}{phantom}
\newcommand{\anchorColor}{red}
\newcommand{\vertexShape}{circle}
\newcommand{\vertexSize}{5.00thick}
\newcommand{\blobSize}{15.00}
\newcommand{\captionSize}{40pt}

% this ensures new commands in fmf are expanded
\newcommand{\efmf}[1]{
  \begingroup\edef\x{\endgroup\noexpand\fmf{#1}}\x
}
\newcommand{\efmfv}[1]{
  \begingroup\edef\x{\endgroup\noexpand\fmfv{#1}}\x
}

% This facilitate toggling between showing and hiding vertex and edge labels
\newcommand{\showAnchorLabel}[1]{#1}
\newcommand{\hideAnchorLabel}[1]{}
\newcommand{\showVertexLabel}[1]{#1}
\newcommand{\hideVertexLabel}[1]{}
\newcommand{\showEdgeLabel}[1]{#1}
\newcommand{\hideEdgeLabel}[1]{$$}

% Remove label color because it sadly interferes with the efmf and efmfv trick above
\newcommand{\setLabelColor}[1]{}
%\newcommand{\setLabelColor}[1]{\color{#1}}

\begin{document}
\begin{fmffile}{diagram_0_massless_triangle_0}
\fmfcmd{
 style_def my_phantom_arrow expr p =
  draw_phantom_arrow p;
  shrink (\arrowSize);
    cfill (arrow p);
  endshrink;
 enddef;
}
\fmfcmd{
 style_def my_double_phantom_arrow expr p =
  draw_phantom_arrow p;
  shrink (\doubleArrowSize);
    cfill (arrow p);
  endshrink;
 enddef;
}
\hspace{8cm}\setlength{\unitlength}{1pt}\fontsize{\labelSize}{\labelSize*1.2}\selectfont
\begin{fmfgraph*}(1500,1200)
 % ignore: latex
% incoming external vertices
\fmfleft{v0}

% outgoing external vertices
\fmfright{v1,v2}

% incoming half edges
% == Drawing edge 'p1' (e0) with particle scalar_0
\efmf{dashes,width=\lineWidth,fore=\lineColor,tension=\defaultTension,label=\showEdgeLabel{\setLabelColor{\labelColor}p1$|\phi0|p_{1}$},label.dist=\labelDistance}{v0,v3}
% Draw an arrow to indicate orientations of vectors as well
\efmf{my_phantom_arrow,width=\lineWidth,fore=\nonLmbColor,tension=0,label.dist=\labelDistance}{v0,v3}
% Draw a phantom additional tension for external leg
\efmf{phantom,width=\lineWidth,fore=black,tension=\externalLegTension,label.dist=\labelDistance}{v0,v3}

% external outgoing half edges
% == Drawing edge 'p2' (e1) with particle scalar_0
\efmf{dashes,width=\lineWidth,fore=\lineColor,tension=\defaultTension,label=\showEdgeLabel{\setLabelColor{\labelColor}p2$|\phi0|p_{2}$},label.dist=\labelDistance}{v4,v1}
% Draw an arrow to indicate orientations of vectors as well
\efmf{my_phantom_arrow,width=\lineWidth,fore=\nonLmbColor,tension=0,label.dist=\labelDistance}{v4,v1}
% Draw a phantom additional tension for external leg
\efmf{phantom,width=\lineWidth,fore=black,tension=\externalLegTension,label.dist=\labelDistance}{v4,v1}
% == Drawing edge 'p3' (e2) with particle scalar_0
\efmf{dashes,width=\lineWidth,fore=\lineColor,tension=\defaultTension,label=\showEdgeLabel{\setLabelColor{\labelColor}p3$|\phi0|p_{3}|(p_{1}-p_{2})$},label.dist=\labelDistance}{v5,v2}
% Draw an arrow to indicate orientations of vectors as well
\efmf{my_phantom_arrow,width=\lineWidth,fore=\nonLmbColor,tension=0,label.dist=\labelDistance}{v5,v2}
% Draw a phantom additional tension for external leg
\efmf{phantom,width=\lineWidth,fore=black,tension=\externalLegTension,label.dist=\labelDistance}{v5,v2}

% internal edges
% == Drawing edge 'q1' (e3) with particle scalar_0
\efmf{dashes,width=\lineWidth,fore=\lineColor,tension=\defaultTension,label=\showEdgeLabel{\setLabelColor{\labelColor}q1$|\phi0|(k_{0}+p_{2})$},label.dist=\labelDistance}{v3,v4}
% Draw an arrow to indicate orientations of vectors as well
\efmf{my_phantom_arrow,width=\lineWidth,fore=\nonLmbColor,tension=0,label.dist=\labelDistance}{v3,v4}
% == Drawing edge 'q2' (e4) with particle scalar_0
\efmf{dashes,width=\lineWidth,fore=\lineColor,tension=\defaultTension,label=\showEdgeLabel{\setLabelColor{\lmbColor}q2$|\phi0|k_{0}$},label.dist=\labelDistance}{v4,v5}
% Draw an arrow to indicate orientations of vectors as well
\efmf{my_phantom_arrow,width=\lineWidth,fore=\nonLmbColor,tension=0,label.dist=\labelDistance}{v4,v5}
% Draw a colored arrow for this LMB edge
\efmf{my_phantom_arrow,width=\lineWidth,fore=\lmbColor,tension=0,label.dist=\labelDistance}{v4,v5}
% == Drawing edge 'q3' (e5) with particle scalar_0
\efmf{dashes,width=\lineWidth,fore=\lineColor,tension=\defaultTension,label=\showEdgeLabel{\setLabelColor{\labelColor}q3$|\phi0|(k_{0}-p_{1}+p_{2})$},label.dist=\labelDistance}{v5,v3}
% Draw an arrow to indicate orientations of vectors as well
\efmf{my_phantom_arrow,width=\lineWidth,fore=\nonLmbColor,tension=0,label.dist=\labelDistance}{v5,v3}

% label vertices
\showVertexLabel{\fmflabel{v4}{v3}}
\efmfv{decor.shape=\vertexShape,decor.filled=full,label.dist=\labelDistance,decor.size=\vertexSize}{v3}
\showVertexLabel{\fmflabel{v5}{v4}}
\efmfv{decor.shape=\vertexShape,decor.filled=full,label.dist=\labelDistance,decor.size=\vertexSize}{v4}
\showVertexLabel{\fmflabel{v6}{v5}}
\efmfv{decor.shape=\vertexShape,decor.filled=full,label.dist=\labelDistance,decor.size=\vertexSize}{v5}

% external tension phantom edges


\end{fmfgraph*}
\end{fmffile}
\begin{center}
\vspace{0.5cm}
{\fontfamily{qcr}\selectfont\fontsize{\captionSize}{\captionSize*1.2}\selectfont Amplitude massless triangle {\bf \fontfamily{qcr}\selectfont\fontsize{\captionSize}{\captionSize}\selectfont \#0}}
\end{center}
\end{document}
